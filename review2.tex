\documentclass{article}
\title{Review: Sholay}
\author{Sandeep Kumar,Harsh Narang,Akshay Wadhwani}

\begin{document}
\maketitle
This is the movie with the greatest cast ever assembled in the whole of  Indian Cinema. Sholay did for Indian Cinema what Casablanca did for Hollywood. It became an instant classic and broke all the rules of film making and characterization, while preserving the values of Indian cinema. It has excellent dialogues, excellent music,excellent Acting and one of the best plots in the history of Indian Cinema.\\
In a 2005 BBC review, the well-rounded characters and simple narrative of the film were commended, but the comical cameos of Asrani and Jagdeep were considered unnecessary. On the film's 35th anniversary, the Hindustan Times wrote that it was a "trailblazer in terms of camera work as well as music," and that "practically every scene, dialogue or even a small character was a highlight." In 2006, The Film Society of Lincoln Center described Sholay as "an extraordinary and utterly seamless blend of adventure, comedy, music and dance", labelling it an "indisputable classic". Chicago Review critic Ted Shen criticised the film in 2002 for its formulaic plot and "slapdash" cinematography, and noted that the film "alternates between slapstick and melodrama".In their obituary of the producer G.P. Sippy, the New York Times said that Sholay "revolutionized Hindi filmmaking and brought true professionalism to Indian script writing"

\end{document}
